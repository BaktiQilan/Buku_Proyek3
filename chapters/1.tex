\section{Bank Sampah}
Pertumbuhan ekonomi masyarakat Indonesia dari tahun ke tahun semakin meningkat diikuti dengan pertumbuhan penduduk. Hal tersebut semakin terasa dampaknya terhadap lingkungan yaitu manusia cenderung merusak lingkungan demi mempertahankan hidupnya. Kualitas lingkungan secara terus menerus semakin menurun sehingga menimbulkan permasalahan degradasi lingkungan pada kehidupan masyarakat. Salah satu permasalahan lingkungan yang masih menjadi problematika di perkotaan yaitu pengelolaan sampah.

Pengelolaan sampah adalah kegiatan yang sistematis,menyeluruh, dan berkesinambungan yang meliputi pengurangandan penanganan sampah. Sehingga pengelolaan pada kawasan perkotaan, dewasa ini dihadapkan kepada berbagai permasalahan yang cukup kompleks. Permasalahan-permasalahan tersebut meliputi tingginya laju timbunan sampah, kepedulian masyarakat (human behaviour) yang masih sangat rendah serta masalah pada kegiatan pembuangan akhir sampah (final disposal) 
Kurangnya kepedulian masyarakat dalam menjaga kebersihan sampah dapat merugikan banyak pihak baik itu merugikan dirinya sendiri maupun masyarakat sekitar. Sampah yang sering kali menumpuk di suatu tempat dan lama kelamaan akan semakin banyak  sehingga menyebabkan banyak masalah, dan hal itu perlu adanya kepedulian dari masyarakat sekitarnya dan sebuah sistem yang dapat mengelola sampah menjadi sesuatu yang berguna.

Selain hal tersebut di dalam masyarakat perkotaan terdapat budaya konsumtif yang mempengaruhi dalam peningkatan kualitas dan jenis sampah. Sehingga dalam pengelolaan sampah tidak akan dapat dipisahkan dengan campur tangan negara dan berbagai sektor yang ada di dalam masyarakat termasuk dunia usaha. Selain itu peran dari masayarakat yang merupakan jejaring atau komunitas pembuang sampah juga mempunyai andil besar dalam pengelolaan sampah dalam hal ini adalah proses daur ulang untuk dapat dimanfaatkan kembali. Sehingga dalam pengelolaan sampah merupakan bagian dari pelayanan publik yang harus diatur dalam regulasi yang diharapkan akan memberikan kenyamanan di dalam kehidupan masyarakat warga sehari-hari.

Di Indonesia sebenarnya terdapat beberapa peraturan perundang-undangan yang mempunyai korelasi maupun berkaitan langsung dengan pengelolaan sampah yaitu Undang-Undang No. 32 Tahun 2009 tentang Perlindungan dan Pengelolaan Lingkungan Hidup, Undang-undang Nomor 32 tahun 2004 tentang Pemerintahan Daerah diganti dengan UU No. 23 Tahun 2014 tentang Pemerintahan Daerah, UU No. 18 Tahun 2008 tentang Pengelolaan Sampah dan beberapa peraturan daerah yang sudah dibentuk oleh pemerintah daerah baik di tingkat Kabupaten atau Kota seperti di Peraturan Daerah Kota Surakarta No. 3 tahun 2010 Tentang Pengelolaan Sampah. Sanksi-sanksi yang terdapat dalam peraturan terutama yang menyangkut pengelolaan sampah tidak memberikan efek jera bagi masyarakat yang tidak melakukan pengelolaan sampah dengan berwawasan lingkungan sehingga perlu dikaji mengenai efektifitas sanksi dalam penegakan hukum dalam pengelolaan sampah. Selain itu peran pemerintah daerah juga sangat penting dalam mengeluarkan kebijakan terhadap pengelolaan sampah. Apabila daerah mampu mengelola sampahnya dengan baik maka pelaksanaan terhadap prinsip Good Environmental Governance sudah dapat dikatakan terpenuhi.Oleh karena itu, akan dirancang sebuah Aplikasi pengelolaan sampah yang akan diintegrasikan dengan Aplikasi Bank Sampah yang akan dirancang pada proyek ini. 

Dalam perancangan aplikasi ini menggunakan sebuah kerangka kerja yaitu codeig-niter yang sudah banyak dipakai dalam pengembangan website, yang ringan dan cepat dalam membangun pemrograman website dinamis. Berbeda dengan cms yang memang dapat di unduh secara gratis namun kurang dapat dikembangkan karena masih bersifat statis sehingga perancangan website Aplikasi Bank Sampah  ini menggunakan kerangka kerja yang dapat dikembangkan dengan dinamis.

\subsection{Bank}
Bank berasal dari bahasa Prancis “Banque” atau dalam bahasa Italia disebut Bianco yang berarti peti, meja atau tempat menyimpan uang. Kata bank dalam bahasa Italia yang berarti meja memang diambil dari kata tersebut karena transaksi keuangan dalam lembaga tersebut biasa dilakukan diatas meja. Dalam  berdasarkan undang-undang yang ada di Indonesia, bank memiliki maknanya tersendiri. Menurut  Undang-undang Nomor 7 Tahun 1992 Tentang Perbankan , Bank adalah badan usaha yang menghimpun dana dari masyarakat dalam bentuk simpanan, dan menyalurkan kepada masyarakat dalam rangka meningkatkan taraf hidup rakyat banyak.

Pengertian Bank Yang dinyatakan Oleh Drs. H. Malayu S.P. Hasibuan, Yakni :
Bank adalah badan usaha kekayaan terutama didalam bentuk aset keuangan (financial assets) dan juga bermotifkan profit serta sosial, jadi bukan hanya mencari keuntungan saja. Bank ialah pencipta dan juga pengedar uang kartal. Pencipta serta pengedar uang kartal (uang kertas dan juga logam) merupakan otoritas tunggal dari bank sentral (Bank Indonesia), sedangkan uang giral dapat diciptakan dengan bank umum.

Jenis jenis bank, maka dilihat dari fungsinya jenis jenis bank ada 4 yaitu :

\begin{enumerate}
	\item Bank Sentral, yakni jenis bank yang bertugas untuk menerbitkan uang kertas dan juga uang logam untuk dapat dijadikan sebagai alat pembayaran yang sah di dalam suatu negara dan juga mempertahankan konversi uang yang dimaksud terhadap emas maupun perak maupun keduanya.
	\item Bank Umum, yakni jenis bank yang bukan saja dapat untuk meminjamkan ataupun menginvestasikan berbagai jenis tabungan yang diperolehnya, namun tetapi juga dapat memberikan pinjaman dari menciptakan sendiri suatu uang giral.
	\item Bank Perkreditan Rakyat (BPR), yaitu jenis bank yang melaksanakan kegiatan usaha dengan secara konvensional maupun yang didasarkan pada suatu prinsip syariah yang dalam kegiatannya tidak dapat memberikan jasa di dalam lalu lintas pembayaran.
	\item Bank Syariah, yakni jenis bank yang  beroperasi dengan berdasarkan prinsip bagi hasil maupun sesuai dengan kaidah ajaran islam mengenai hukum riba.
\end{enumerate}

\subsection{Sampah}
Sampah adalah tumpukan bahan bekas dan sisa tanaman (daun, sisa sayuran, sisa buangan lain), atau sisa kotoran hewan atau benda-benda lain yang dibuang. Dalam pengertian yang luas, sampah diartikan sebagai benda yang dibuang, baik yang berasal dari alam ataupun dari hasil proses teknologi (Reksosoebroto, 1990). Menurut Wasito (1970) sampah ialah segala zat padat atau semi padat yang terbuang atau yang sudah tidak berguna, baik yang dapat membusuk atau yang tidak dapat membusuk kecuali zat-zat buangan atau kotoran yang keluar dari tubuh manusia (kotoran atau najis manusia).

Menurut Reksosoebroto (1990), bahwa penanganan sampah yang baik akan memberikan manfaat yang besar bagi kehidupan manusia dan lingkungan. Manfaat lain penanganan sampah yang baik adalah menurunkan 90\% angka kehidupan lalat menurunkan 90\% angka kehidupan tikus menurunkan 30\% angka kehidupan nyamuk, menurunkan 70\% angka kerusakan jembatan dan menurunkan 90\% angka kerusakan pipa bangunan. Keuntungan pembuangan sampah yang dapat diperoleh dari pengelolaan sampah yang baik dapat dilihat dari beberapa segi yaitu: (1) Dari segi sanitasi, menjamin tempat kerja yang bersih, mencegah tempat berkembang biaknya vektor hama penyakit dan mencegah pencemaran lingkungan termasuk timbulnya pengotoran sumber air; (2) Dari segi ekonomi mengurangi biaya perawatan dan pengobatan sebagai akibat yang ditimbulkan sampah. Tempat kerja yang bersih akan meningkatkan gairah kerja dan akan menambah produktivitas serta efisiensi pekerja, menarik banyak tamu atau pengunjung, mengurangi kerusakan sehingga mengurangi biaya perbaikan (3) Dari segi estetika, menghilangkan pemandangan tidak sedap dipandang mata menghilangkan timbulnya bau–bauan yang tidak enak, mencegah keadaan lingkungan yang kotor dan tercemar. Penanganan sampah yang baik akan memberikan manfaat yang besar bagi kehidupan manusia dan lingkungan.

\subsection{Pengertian Bank Sampah}
Menurut peraturan menteri negara lingkungan hidup RI nomor 13 tahun 2012 tentang pedoman pelaksanaan \textit{Reduce}, \textit{Reuse}, dan \textit{Recycle} melalui bank sampah, yang di atur dalam pasa 1 ayat 2 dengan bunyi yaitu: "Bank sampah adalah tempat pemilahan dan pengumpulan sampah yang dapat didaur ulang dan/atau diguna ulang yang memiliki nilai ekonomi."

Bank sampah adalah suatu tempat yang digunakan untuk mengumpulkan sampah yang sudah di \textit{filter}. Hasil dari pengemupulan tersebut akan disetorkan ke tps(untuk sampah yang tidak dapat digunakan kembali) atau ke tempat pembuatan kerajinan dari barang-barang bekas. Bank sampah dikelola menggunakan sistem seperti perbankkan yang dilakukan oleh petugas. Penyetor atau nasabah ialah warga masyarakat yang berada di sekitar lokasi bank yang dimana nasabah akan mendapatkan rekening seperti halnya rekening pada bank

\subsection{Klasifikasi Sampah}
Sampah rumah tangga, secara umum terklasifikasi menjadi 2 jenis, yaitu sampah organik dan sampah non-organik. Sampah organik adalah sampah yang berasal dari makhluk hidup dan dapat terurai kembali oleh alam, sedangkan sampah non-organik adalah sampah yang berasal dari bahan hasil olahan manusia.

\subsection{Tujuan dan Manfaat Bank Sampah}
Tujuan dibangunnya bank sampah sebenarnya bukan bank sampah itu sendiri. Bank sampah adalah strategi untuk membangun kepedulian masyarakat agar dapat berka-wan dengan sampah untuk mendapatkan manfaat ekonomi langsung dari sampah. Jadi, bank sampah tidak dapat berdiri sendiri melainkan harus diintegrasikan dengan gerakan 4R sehingga manfaat langsung yang dirasakan tidak hanya ekonomi, namun pembangunan lingkungan yang bersih, hijau dan sehat.

Bank sampah juga dapat dijadikan solusi untuk mencapai pemukiman yang bersih dan nyaman bagi warganya. Dengan pola ini maka warga selain menjadi disiplin dalam mengelola sampah juga mendapatkan tambahan pemasukan dari sampah- sampah yang mereka kumpulkan. Tampaknya pemikiran seperti itu pula yang ditangkap oleh Kementerian Lingkungan Hidup. September lalu instansi pemerintah ini menargetkan membangun bank sampah di 250 kota di seluruh Indonesia. Menteri Negara Lingkungan Hidup Balthasar Kambuaya mengatakan sampah sudah menjadi ancaman yang serius, bila tidak dikelola dengan baik. Bukan tidak mungkin beberapa tahun mendatang sekitar 250 juta rakyat Indonesia akan hidup bersama tumpukan sampah di lingkungannya.


\subsection{Prinsip dan Mekanisme Bank Sampah}
\subsubsection{Prinsip}
\hfill\\
Sebenarnya semua orang sudah pernah, dalam kehidupan sehari-hari, melihat cara kerja sebuah bank sampah. Hanya saja, mayoritas tidak menyadari bahwa hal yang sama diterapkan dalam pengelolaan bank sampah hingga menghasilkan uang.

Dalam sistem kerja sebuah bank sampah, akan ada 3 pihak yang terlibat.
\begin{enumerate}
	\item Penyetor : Masyarakat alias nasabah yang merupakan sumber sampah yang akan dikelola oleh sebuah bank sampah. Biasanya perorangan.

\item Bank Sampah : Kelompok yang bertugas menerima dan kemudian mengolah sampah dari penyetor dan menjualnya kepada pihak-pihak yang bisa memanfaatkan sampah

\item Pembeli : Mereka yang membeli sampah yang dikelola oleh sebuah bank sampah. Bisa perorangan dan bisa juga sebuah perusahaan.
\end{enumerate}

\subsubsection{Mekanisme}
\hfill\\
Pengelolaan sampah berbasis bank memberikan banyak manfaat bagi masyarakat. Keuntungan berupa kebersihan lingkungan, kesehatan, hingga ekonomi. Berikut mekanisme kerja bank sampah.
\begin{enumerate}
\item Pemilahan sampah rumah tangga

Nasabah harus memilah sampah sebelum disetorkan ke Bank Sampah. Pemilahan sampah tergantung pada kesepakatan saat pembentukan bank sampah. Misalnya, berdasarkan kategori sampah organik dan anorganik. Biasanya, sampah anorganik kemudian dipisahkan lagi berdasarkan jenis bahan : plastik, kertas, kaca, dan lain-lain. Pengelompokan sampah akan memudahkan proses penyaluran sampah. Apakah akan disampaikan ke tempat pembuatan kompos, panrik plastik atau industri rumah tangga.

Dengan sistem bank sampah, masyarakat secara tidang langsung telah membantu mengurangi timbunan sampah di tempat pembuangan akhir. Sebab, sebagian sampah yang telah dipilah dan dikirimkan ke bank sampah akan dimanfaatkan kembali, sehingga yang tersisa dan dibuang ke TPA, hanya sampah yang tidak dapat bernilai ekonomi.

\item Penyetoran Sampah ke Bank Sampah

Waktu penyetoran sampah biasanya telah disepakati sebelumnya. Misalnya, dua hari dalam sepekan setiap Rabu dan Sabtu. Penjadwalan ini maksudnya untuk menyamakan waktu nasabah menyetor dan pengangkutan ke pengepul. Hal ini agar sampah tidak tertumpuk di lokasi bank sampah.

\item Penimbangan

Sampah yang sudah dusetor ke bank kemudian ditimbang. Berat sampah yang bisa disetorkan sudah ditentukan pada kesepakatan sebelumnya, misalnya minimal harus satu kilogram.

\item Pencatatan

Petugas akan mencatat jenis dan bobot sampah setelah penimbangan. Hasil penimbangan tersebut lalu dikonversikan ke dalam nilai rupiah yang kemudian ditulis dibuku tabungan. Pada sistembank sampah, tabungan biasanya bisa diambil setiap tiga bulan sekali. Tabungan bank sampah bisa dimodifikasi menjadi beberapa jenis : tabungan hari raya, tabungan pendidikan dan tabungan yang bersifat sosial untuk disalurkan melalui lembaga kemasyarakatan. 

Pada tahap ini, nasabah akan merasakan keuntungan sistem bank sampah. Dengan menyisihkan sedikit tenaga untuk memilah sampah, masyarakat akan mendapatkan keuntungan berupa uang tabungan. Dengan sistempengelolaan sampah yang "konvensional", masyarakat justru harus mengeluarkan uang, membayar petugas kebersihan untuk mengelola sampahnya.

\item Pengangkutan

Bank sampah sudah bekerja sama dengan pengepul yang sudah ditunjuk dan disepakati. Sehingga setelah sampah terkumpul, ditimbang dan dicatat langsung diangkat ke tempat pengolahan sampah berikutnya. Jadi, sampah tidak menumpuk di lokasi bank sampah.

Bank sampah bisa berkembang menjadi sumber bahan baku untuk industri rumah tangga di sekitar lokasi bank. Jadi, pengolahan sampah bisa dilakukan oleh masyarakat yang juga menjadi nasabah bank. Sehingga masyarakat bisa mendapat keuntugan ganda dari sistem bank sampah yaitu tabungan dan laba dari hasil penjualan produk dari bahan daur ulang.
\end{enumerate}

\subsection{Pendirian dan pengembangan sistem bank sampah}
\begin{enumerate}
\item Sosialisai Awal

Sosialisasi awal dilakukan untuk memberikan dasar mengenai bank sampah kepada masyarakat. Wacana disampaikan antara lain tentang bank sampah sebagai program nasional, pengertian bank sampah. Penjelasan harus menonjolkan berbagai sisi positif sistem bank sampah. Sehingga warga tergerak untuk melaksanakan sistem bank sampah.

\item Pelatihan Teknis

Setelah warga sepakat untuk melaksanakan sistem bank sampah. maka perlu dilakukan pertemuan lanjutan. Tujuannya untuk memberi penjelasan detail tentang standarisai sistem bank sampah. mekanisme kerja bank sampah dan keuntungan sistem bank sampah. Sehingga warga menjadi lebih siap pada saat harus melakukan pemilihan sampah hingga penyetoran ke bank. Forum tersebut juga harus dimanfaatkan untuk musyawarah penentuan nama bank sampah, pengurus, lokasi kantor dan tempat penimbangan, pengepul hingga jadwal penyetoran sampah

\item Pelaksanaan Sistem Bank Sampah

Pelaksanaan bank sampah dilakukan pada saat hari yang telah di sepakati. Pengurus siap dengan keperluan administrasi dan peralatan timbang. Nasabah administrasi dan peralatan timbang. Nasabah datang ke kantor bank dan lokasi penimbangan dengan membawa sampah yang sudah dipilah. Nasabah akan mendapat uang yang disimpan dalam bentuk tabungan sesuai dengan nilai sampah yang disetor.

\item  Pemantauan dan Evaluasi

Berbagai tantangan mungkin muncul saat penerapan bank sampah. Organisasi masyarakat harus tetap melakukan pendampingan selama sistem berjalan. Sehingga bisa membantu warga untuk memecahkan masalah dengan cepat. Evaluasu dilakukan untuk pelaksanaan bank sampah yang lebih baik.

\item Pengembangan

Sistem bank sampah bisa berkembang menjadi unit simpan pinjam, unit usaha sembako, koperasi dan pinjaman modal usaha. Perluasan fungsi bank sampah ini bisa disesuaikan dengan kebutuhan masyarakat. Misalnya, jika kebanyakn warga adalah wirausaha, pengembang bank sampah diarahkan untuk unit pinjam modal usaha. Salah sati bentuk bantuan dari organisasi masyarakat pada proses ini antara lain dalam pengurusan badan hukum koperasi.
\end{enumerate}

