\section{Pengenalan Bank Sampah}
Bank sampah ialah suatu terobosan yang berfokus pada pengelolaan sampah rumah tangga. Dimana sampah rumah tangga tersebut mampu di kelola dan memiliki timbal balik yang menguntungkan baik bagi rumah tangga yang mengeluarkan sampah, maupun pemerintah atau pengelola sampah. dengan hadirnya terobosan bank sampah ini diharapkan sampah-sampah rumah tangga dapat menguntungkan berbagai pihak. 

Berdirinya bank sampah ini terpicu karena keprihatinan dan ketidaknyamanan masyarakat terhadap lingkunganya yang dari hari ke hari makin banyak sampah yang menumpuk, dikarenakan pembersihan yang dilakukan lebih sedikit dengan sampah yang dihasilkan, terutama dalam sampah rumah tangga. Oleh karena itu bank sampah hadir dengan maksud untuk memilah dan memilih(yang masih bisa digunakan, di daur ulang, atau masih bermanfaat) sampah-sampah rumah tangga tersebut.

\section{Definisi Bank Sampah}
Bank sampah adalah suatu tempat yang digunakan untuk mengumpulkan sampah yang sudah di \textit{filter}. Hasil dari pengemupulan tersebut akan disetorkan ke tps(untuk sampah yang tidak dapat digunakan kembali) atau ke tempat pembuatan kerajinan dari barang-barang bekas. Bank sampah dikelola menggunakan sistem seperti perbankkan yang dilakukan oleh petugas. Penyetor atau nasabah ialah warga masyarakat yang berada di sekitar lokasi bank yang dimana nasabah akan mendapatkan rekening seperti halnya rekening pada bank.\cite{1}