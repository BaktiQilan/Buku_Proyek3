\section{Coba}
BAB II
LANDASAN TEORI
2.1	Tinjauan Pustaka
2.1.1	Sistem
Sistem adalah suatu kelompok dari komponen-komponen yang saling berhubungan dengan satu sama lain dengan tujuan yang sama untuk mencapai tujuan tertentu. Sedangkan Informasi adalah berupa data yang diolah menjadi bentuk lebih berguna dan berarti bagi pemakainya [1]. Sehingga dapat disimpulakan bahwa system informasi adalah suatu komponen yang saling berhubungan yang bertujuan mengumpulkan,mengubah,dan menyebarkan informasi dalam sebuah organisasi.
2.1.2	Pengendalian
Apikasi adalah penggunaan dalam suatu computer, intruksi (instruction) atau pernyataan (statement) yang disusun sedemikian rupa sehingga computer dapat memproses input menjadi output. Aplikasi dapat diartikan juga sebagai program yang dibuat untuk menolong manusia dalam melakukan tugas tertentu . [2] 
Berdasarkan pernyataan diatas, dapat diambil kesimpulan bahwa aplikasi adalah suatu program computer yang dibuat sedemikian rupa dengan suatu fungsi bagi pengguna jasa aplikasi yang dapat memproses input dan output oleh suatu sasaran yang akan dituju.
2.1.3	Bank Sampah
Bank sampah adalah suatu tempat yang digunakan untuk mengumpulkan
sampah yang sudah dipilah-pilah. Sampah yang dikumpulkan adalah sampah
yang mempunyai nilai ekonomis. Cara kerja bank sampah umumnya hampir
Sama dengan bank lainnya, ada nasabah, pencatatan pembukuan, dan manajemen pengelolaannya, apabila dalam bank yang biasa kita kenal yang disetorkan nasabah adalah uang akan tetapi dalam bank sampah yang disetorkan adalah sampah yang mempunyai nilai ekonomis. Sedangkan pengelola bank sampah harus orang kreatif dan inovatif serta memiliki jiwa kewirausahaan agar dapat meningkatkan pendapatan masyarakat. Sistem kerja bank sampah pengelolaan sampahnya berbasis rumah tangga, dengan memberi reward kepada yang berhasil memilah dan menyetorkan sejumlah sampah. (Clean (Ulfah, 2016))[7]
2.1.4	Website
Website adalah sebuah situs yang mana sekumpulan halaman yang menampilkan informasi data teks, data gambar diam atau gerak, data animasi, suara, video atau gabungan dari semuanya, baik yang bersifat statis atau dinamis yang membentuk satu rangkaian bangunan yang saling terkait dimana masing-masing dihubungkan dengan jaringan-jaringan halaman (hyperlink). [3]
Berdasarkan pengertian diatas dapat kita ketahui bahwa web adalah sekumpulan halaman yang di dalamnya terdapat multimedia yang saling berhubungan, dan dapat diakses melalui koneksi jaringan.
2.1.5	Codeigniter
CodeIgniter adalah kerangka kerja aplikasi web untuk membangun aplikasi web yang sifatnya open source yang digunakan untuk membangun aplikasi PHP yang dinamis. CodeIgniter memiliki Karakteristik yang sangat fleksibel dan ringan dalam memodifikasi dan mengintegrasikan suatu website. Dalam CodeIgniter dapat juga digunakan pola Model View Controller (MVC) sehingga struktur kode yang di hasilkan lebih terstruktur dan memiliki standar yang jelas. [4]
2.1.6	PHP
PHP adalah singkatan dari Hypertext Preprocessor yang merupakan bahasa pemrograman yang bersifat open source yang berada didalam server yang diproses di server. PHP salah satu bahasa pemrograman yang memiliki script yang terintegrasi dengan HTML dan berada pada server (server side HTML embedded sripting). [5]
PHP bisa berinteraksi dengan database, file dan folder, PHP ini termaksud bahasa cross-platform, yang mana PHP ini bisa berjalan disistem operasi yang berbeda-beda (Windows, Linux, ataupun MAC).[5]
Berdasarkan pernyataan diatas dapat disimpulkan bahwa PHP adalah bahasa adalah sebuah bahaha pemrograman yang sangat fleksibel untuk digunakan dalam membuat sebuah website, dan juga PHP dapat dijalankan di bawah sistem operasi LINUX dan Windows.
2.1.7	MySQL (My Structure Query Language)
MySQL adalah sebuah server database open source yang terkenal yang digunakan berbagai aplikasi terutama untuk server atau membuat web. MySQL juga adalah sebuah implementasi dari sistem manajemen basis data relasional (RDBMS). Kehandalan suatu sistem basis data (DBMS) dapat diketahui dari cara kerja pengoptimasinya dalam melakukan proses perintah-perintah SQL yang dibuat oleh pengguna maupun program-program aplikasi yang memanfaatkannnya. 
		Berdasarkan dari pengertian diatas kesimpulan dari MySQL adalah sekumpulan sistem database yang banyak digunakan untuk pengembangan suatu aplikasi berbasis web dan bersifat open source.
2.1.8	UML (Unified Modeling Language)
The Unified Modeling Language (UML) adalah standar de facto untuk analisis perangkat lunak berorientasi objek dan pemodelan desain. [9] UML membantu memodelkan berbagai aspek sistem melalui berbagai diagram yang di dukungnya. Setiap aspek dari suatu sistem disajikan dengan menggunakan jenis diagram UML tertentu dan satu set diagrama disebut sebagai model.
2.1.9	Database 
Database didefinisikan sebagai kumpulan data yang saling berkaitan secara teknis, database juga sebagai tempat media penyimpanan data kita dalam membuat sebuah program yang berisikan tabel, field, record, yang diselimuti dengan DBMS (Database Management System). [7]
Berdasarkan dari uraian di atas dapat disimpulkan bahwa database adalah sekumpulan data yang beris tabel, field, dan record yang disimpan secara sistematis.
BAB III
ANALISIS DAN PERANCANGAN
3.1	Analisis
Analisis merupakan langkah awal untuk pengembangan sebuah aplikasi, karena perancangan dan bahkan pengembangan implementasi aplikasi tidak akan berjalan dengan baik tanpa adanya analisa terhadap aplikasi yang akan digunakan. Analisis juga dapat didefinisikan sebagai penguraian dari suatu sistem informasi yang utuh kedalam bagian-bagian komponennya dengan maksud untuk mengidentifikasi dan mengevaluasi masalah-masalah, kesempatan-kesempatan, hambatan-hambatan yang terjadi serta kebutuhan yang diharapkan sehingga dapat diusulkan perbaikan agar mendapat hasil yang maksimal. 
Analisis yang dilakukan terhadap Aplikasi Bank Sampah Menggunakan Codelgniter ini dibuat menggunakan flowmap dan metode Object Oriented yang memberikan gambaran mengenai proses yang terdapat di dalam aplikasi tersebut. 
Sistem ini dibangun dengan menggunakan model MVC atau Model View Controller, dimana MVC merupakan  model pengembangan dari Object Oriented Programming, dimana setiap baris kodenya dipisahkan menjadi tiga bagian, yaitu ada pada view sebagai form, lalu controller untuk menyimpan fungsi dan class dan model untuk menyimpan database.
3.1.1	Analisis sistem yang sedang berjalan
Sistem yang berjalan saat ini terdiri dari satu prosedur yaitu alur pemasukan barang yang sedang berjalan.
 
Gambar 3.1 Prosedur yang sedang berjalan
Keterangan :
1.	Masyarakat membuang sampah pada TPS yang sudah tersedia di depan rumahnya masing-masing
2.	Masyarakat Melakukan pembayaran ke kepala desa setiap sebulan sekali
3.	Kepala desa mengkonfirmasi ke pada petugas kebersihan untuk melakukan pengambilan sampah setiap seminggu 2 kali.
4.	Petugas kebersiahan akan mengambil sampah setiap 2 kali dalam seminggu.
5.	Petugas kebersiahan akan mengelola sampah, sampah apa saja yang akan di daur ulang dan sampah apa saja yang akan di bakar.

3.1.2	Analisis Sistem yang akan di bangun
Pada analisis sistem yang akan dibangun ini, dibuat beberpa pembaruan dari yang  sebelumnya. Pada prosedur ini masyarakat di haruskan mendaftarkan terlebih dahulu untuk menjadi nasabah pada Bank Sampah. Prosedur yang akan dibangun pada Bank Sampah yaitu sebagai berikut :
 
Gambar 3.2 Prosedur yang akan dibangun
Keterangan :
1.	Masyarakat harus mendaftarkan diri terlebih dahulu untuk menjadi nasabah pada bank sampah
2.	Setelah menjadi nasabah, masyarakat dapat menabung sampah pada bank sampah
3.	Jika masyarakat menabung sampah, maka data-data sampah tersebut akan di kelola oleh admin 
4.	Admin akan mengkonfirmasi kepada petugas untuk melakukan pengambilan sampah pada nasbah yang telah menabung sampah
5.	Petugas akan diberi alamat nasabah oleh admin
6.	Petugas melakukan pengambilan sampah ke alamat nasabah

3.1.3	Analisis Sistem Login yang akan di bangun
Pada analisis sistem yang akan dibangun ini, dibuat beberpa pembaruan dari yang  sebelumnya. Pada prosedur ini masyarakat di haruskan mendaftarkan terlebih dahulu untuk menjadi nasabah pada Bank Sampah. Prosedur login yang akan dibangun pada Bank Sampah yaitu sebagai berikut :
 
Gambar 3.3 Prosedur Login
Keterangan :	
1.	User akan masuk ke halaman login 
2.	JIka sudah memiliki akun maka user akan langsung melakukan login dan masuk ke halaman user
3.	Jika belum memiliki akun maka user akan melakukan registrasi terlebih dahulu 
4.	User disini yaitu nasabah atau petugas

3.1.3.1	Kebutuhan Fungsional (Functional Requirements)
Kebutuhan fungsional adalah jenis kebutuhan yang berisi proses-proses apa saja yang nantinya dilakukan oleh sistem. Kebutuhan fungsional juga berisi informasi-informasi apa saja yang harus ada dan dihasilkan oleh sistem.
Adapun kebutuhan fungsional yang akan dibuat yaitu:
1.	Login
2.	Kelola data user
3.	Kelola data sampah
4.	Menabung sampah
5.	Pengambilan sampah
Setiap proses memiliki respresentasi masing-masing pada sebuah tabel atau data yang terdapat pada database yang telah dirancang sebelumnya. Dan setiap proses berhubungan langsung dengan entitas atau user.
3.1.3.2	Kebuthan Non-Fungsional (Non-Functional Requirement)
Analisis kebutuhan non fungsional dilakukan untuk mengetahui spesifikasi kebutuhan untuk sistem. Spesifikasi kebutuhan melibatkan analisis perangkat keras/hardware, analisis perangkat lunak/software, analisis pengguna/user.
Adapun kebutuhan non fungsional yang akan dibuat adalah sebagai berikut :
A.	Kebutuhan Perangkat Keras
Pembuatan aplikasi ini menggunakan perangkat sebagai berikut :
Tabel 3.1 Kebutuhn perangkat keras
No.	Jenis	 	Keterangan
1	Processor	:	Intel® core™i3 
2	Memory	:	4 GB
3	Monitor	:	LCD 14,1 Inchi
4	Mouse dan keyboard	:	Standard
B.	Kebutuhan Perangkat Lunak
Spesifikasi perangkat lunak yang digunakan adalah sebagai berikut :

Tabel 3. 2 Kebutuhan perangkat lunak
No.	Jenis	Keterangan
1.	Sistem Operasi	Windows 10 Pro 64-Bit
2.	Server Database	Xampp 1.8.1
3.	Bahasa Pemrograman	PHP dan Android
4.	Software Pendukung	Visual Studio Code dan Android Studio 
5.	Browser	Google Chrome

C.	Analisis Pengguna/User
Aplikasi yang akan dibuat ini digunakan ketika ingin menabung sampah dan melakukan penjemputan sampah, adapun User yang dilibatkan antara lain : Nasabah (Masyarakat) dan petugas kebersihan.
3.2	Perancangan
3.2.1	Use Case Diagram
Use case diagram mendeskripsikan sebuah interaksi antara satu atau lebih actor dengan sistem informasi yang akan dibuat. Use case diagram digunakan untuk mengetahui fungsi apa saja yang ada di dalam sebuah sistem informasi dan siapa saja yang berhak menggunakan fungsi-fungi itu.
3.2.1.1	Definisi Aktor
Pada bagian ini akan dideskripsikan actor-aktor yang terlibat dalam Aplikasi Bank Sampah
Tabel 3.3 Definisi Aktor
No	Aktor	Deskripsi
1.	Admin	-	Mengelola data sampah
-	Mengelola data user
-	Mengatur jadwal pengambilan sampah
2.
	Nasabah	-	Menabung sampah 
3.	Petugas	-	Kelola data sampah

Definisi Use Case
Use case digunakan untuk mengetahui fungsi apa aja yang ada didalam sebuah sistem informasi dan siapa saja yang berhak menggunakan fungsi-fungsi itu.
Tabel 3.4 Definisi Use Case
